\section{Introduction}  
The \emph{ABout Time Linux (AbTLinux)} package managers design will be dealt with in
this document. I will cover them in the following sections:
 
\begin{enumerate}
	\item \textbf{Use Cases} - clearly defined requirements.
	\item \textbf{Scenarios} - use cases worked out in clear examples.
	\item \textbf{Design} - details of \emph{AbTLinux} design, worked out in
		\emph{UML} diagrams.
\end{enumerate}

The driving force behind this project is to design a basic software
package management system that will provide the needed infrastructure to not
only install software packages, but also be able to maintain the systems it is
installing the software packages onto. I believe this can be done using
source based packages and by avoiding the existing RPM and DEB package
management systems. This is not new, nor do I pretend to have found the Holy
Grail of package management tools. The primary goal is a well documented design, with
clearly documented coding practices which will result in an easily
maintainable package manager. This in turn will be the foundation of this Linux
distribution. Make no mistake about it, the clarity of design and coding
practices will take the foremost priority in this project.

I will present my ideas in the following sections starting with the use cases
that will be used to detail the requirements. These will be followed by
scenarios which provide explicit examples for each requirement. Furthermore, I
will put forth a design for a package management system, which will include
implementation choices such as the language and system requirements.